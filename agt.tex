\documentclass{amsart}

\usepackage{babel}
\usepackage{textcomp}
\usepackage{mathrsfs}
\usepackage{url}
\usepackage{amstext}
\usepackage{amsthm}
\usepackage{amssymb}
\usepackage{stmaryrd}
\usepackage{agt}

\begin{document}

\author{Victor Porton}
\email{porton@narod.ru}

\title[Algebraic General Topology]{Algebraic General Topology summary}

\maketitle

In my book~\cite{PortonVolume1} I introduce some new concepts
generalizing general topology, including \emph{funcoids} and \emph{reloids}.
The book along with supplementary materials (such as a partial draft of the second volume)
is freely available (including the \LaTeX\ source) online.

Before studying funcoids, in my book I consider
co-brouwerian lattices and lattices of filters in particular,
as the theory of funcoids is based on theory of filters.
My book contains probably the best (and most detailed) published overview of properties of filters.

Because you always can refer to my book, in this short intro I present theorems without proofs.

I denote order on a poset as~$\sqsubseteq$ and corresponding lattice operations as~$\bigsqcup$ and~$\bigsqcap$.
I denote the least and greatest elements (if they exist) of our poset as~$\bot$ and~$\top$ correspondingly.

\section{Filters}

\begin{defn}
I order the set~$\mathfrak{F}$ of filters (including the improper filter) \emph{reverse} to set-theoretic order, that is
$\mathcal{A} \sqsubseteq \mathcal{B} \Leftrightarrow \mathcal{A} \supseteq \mathcal{B}$
for $\mathcal{A},\mathcal{B}\in\mathfrak{F}$.
\end{defn}

\begin{prop}
This makes the set of filters on a set into a co-brouwerian (and thus distributive) lattice, that is we have
$\mathcal{A} \sqcup \bigsqcap S = \bigsqcap_{\mathcal{X}\in S} (\mathcal{A} \sqcup \mathcal{X})$
for a set~$S$ of filters and a filter~$\mathcal{A}$.
\end{prop}

\section{Funcoids}

Let $\mathfrak{F}(A)$, $\mathfrak{F}(B)$ be sets of filters on sets~$A$,~$B$.
They are complete atomistic co-brouwerian lattices.

\begin{defn}
A \emph{funcoid} $A\rightarrow B$ is a quadruple $(A,B,\alpha,\beta)$
where~$\alpha$ and~$\beta$ are functions $\mathfrak{F}(A)\rightarrow \mathfrak{F}(B)$
and $\mathfrak{F}(B)\rightarrow \mathfrak{F}(A)$ correspondingly, such that
$\mathcal{Y} \sqcap \alpha(\mathcal{X}) \ne \bot \Leftrightarrow \mathcal{X} \sqcap \beta(\mathcal{Y}) \ne \bot$
for every $\mathcal{X}\in\mathfrak{F}(A)$, $\mathcal{Y}\in\mathfrak{F}(B)$.
\end{defn}

\begin{defn}
I denote $(A,B,\alpha,\beta)^{-1} = (B,A,\beta,\alpha)$.
\end{defn}

\begin{defn}
I denote $\langle(A,B,\alpha,\beta)\rangle = \alpha$.
\end{defn}

Funcoids generalize such things as:
\begin{itemize}
\item binary relations;
\item proximity spaces;
\item pretopologies;
\item preclosures.
\end{itemize}

For a proximity~$\delta$, define
\[ \mathcal{X}\mathrel{\delta'}\mathcal{Y} \Leftrightarrow \forall X\in\mathcal{X},Y\in\mathcal{Y}: X\mathrel{\delta}Y \]
for all filters~$\mathcal{X}$,~$\mathcal{Y}$.
Then we have a unique funcoid~$f$ such that
\[
\mathcal{X}\mathrel{\delta'}\mathcal{Y} \Leftrightarrow
\mathcal{Y}\sqcap\langle f\rangle\mathcal{X} \ne \bot \Leftrightarrow
\mathcal{X}\sqcap\langle f^{-1}\rangle\mathcal{Y} \ne \bot.
\]

\begin{defn}
$\mathcal{X} \mathrel{[f]} \mathcal{Y} \Leftrightarrow \mathcal{Y}\sqcap\langle f\rangle\mathcal{X} \ne \bot \Leftrightarrow
\mathcal{X}\sqcap\langle f^{-1}\rangle\mathcal{Y} \ne \bot$.
\end{defn}

\begin{prop}
A funcoid~$f: A\rightarrow B$ is uniquely determined by $\langle f\rangle$ and moreover is uniquely
determined by values of the function~$\langle f\rangle$ on principal filters or
by the relation~$[f]$ between principal filters.
\end{prop}

See my book for formulas for \emph{principal funcoids} that is funcoids corresponding to binary relations
and for funcoids corresponding to pretopologies and preclosures (particularly funcoids corresponding to
topological spaces, as topological spaces can be considered as a special case of either pretopologies or preclosures).

There are also several other equivalent ways to define funcoids.

Funcoids are made more interesting than topological spaces by a new operation (missing in traditional general topology),
\emph{composition}, which is defined by the formula
\[ (B,C,\alpha_2,\beta_2)\circ (A,B,\alpha_1,\beta_1) = (A,C,\alpha_2\circ\alpha_1,\beta_1\circ\beta_2). \]

\section{Reloids}

Reloids are basically just filters on cartesian product~$A\times B$ of two given sets~$A$ and~$B$.

Formally:

\begin{defn}
A \emph{reloid} is a triple $(A,B,F)$ where $A$,~$B$ are sets and $F$ is a filter on~$A\times B$.
\end{defn}

Reloids are a generalization of uniform spaces and of binary relations.

\begin{defn}
Composition $g\circ f$ of reloids can be easily defined as the reloid determined by the filter base
consisting of compositions of binary relations defining these reloids (see the book for an exact formula).
\end{defn}

\begin{prop}
Composition of reloids (and of funcoids) is associative.
\end{prop}

The sets of funcoids and reloids constitute complete atomistic co-brouwerian lattices and these lattices have interesting
properties.

There are interesting relationships between funcoids and reloids, as well as special classes of funcoids and reloids.

\section{Continuity}

A function~$f$ from a space~$\mu$ to a space~$\nu$ is \emph{continuous} iff
$f\circ\mu\sqsubseteq \nu\circ f$. This formula works for continuity, proximal continuity,
uniform continuity, etc., so making all kinds of continuity described by the same formula.

\section{Other}

My book~\cite{PortonVolume1} also considers pointfree generalizations of funcoids,
multidimensional generalizations of funcoids and reloids and other research topics.

I also introduce generalized limit for arbitrary (not necessarily continuous) functions.

My work introduced \emph{many} new conjectures. So I give you a work.

\begin{thebibliography}{9}
	\bibitem{PortonVolume1}
	Victor Porton.
	Algebraic General Topology. Volume~1. \\
	\url{http://www.mathematics21.org/algebraic-general-topology.html}
\end{thebibliography}

\end{document}
